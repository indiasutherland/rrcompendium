% Options for packages loaded elsewhere
\PassOptionsToPackage{unicode}{hyperref}
\PassOptionsToPackage{hyphens}{url}
\PassOptionsToPackage{dvipsnames,svgnames,x11names}{xcolor}
%
\documentclass[
  authoryear,
  preprint,
  3p]{elsarticle}

\usepackage{amsmath,amssymb}
\usepackage{iftex}
\ifPDFTeX
  \usepackage[T1]{fontenc}
  \usepackage[utf8]{inputenc}
  \usepackage{textcomp} % provide euro and other symbols
\else % if luatex or xetex
  \usepackage{unicode-math}
  \defaultfontfeatures{Scale=MatchLowercase}
  \defaultfontfeatures[\rmfamily]{Ligatures=TeX,Scale=1}
\fi
\usepackage{lmodern}
\ifPDFTeX\else  
    % xetex/luatex font selection
\fi
% Use upquote if available, for straight quotes in verbatim environments
\IfFileExists{upquote.sty}{\usepackage{upquote}}{}
\IfFileExists{microtype.sty}{% use microtype if available
  \usepackage[]{microtype}
  \UseMicrotypeSet[protrusion]{basicmath} % disable protrusion for tt fonts
}{}
\makeatletter
\@ifundefined{KOMAClassName}{% if non-KOMA class
  \IfFileExists{parskip.sty}{%
    \usepackage{parskip}
  }{% else
    \setlength{\parindent}{0pt}
    \setlength{\parskip}{6pt plus 2pt minus 1pt}}
}{% if KOMA class
  \KOMAoptions{parskip=half}}
\makeatother
\usepackage{xcolor}
\setlength{\emergencystretch}{3em} % prevent overfull lines
\setcounter{secnumdepth}{5}
% Make \paragraph and \subparagraph free-standing
\makeatletter
\ifx\paragraph\undefined\else
  \let\oldparagraph\paragraph
  \renewcommand{\paragraph}{
    \@ifstar
      \xxxParagraphStar
      \xxxParagraphNoStar
  }
  \newcommand{\xxxParagraphStar}[1]{\oldparagraph*{#1}\mbox{}}
  \newcommand{\xxxParagraphNoStar}[1]{\oldparagraph{#1}\mbox{}}
\fi
\ifx\subparagraph\undefined\else
  \let\oldsubparagraph\subparagraph
  \renewcommand{\subparagraph}{
    \@ifstar
      \xxxSubParagraphStar
      \xxxSubParagraphNoStar
  }
  \newcommand{\xxxSubParagraphStar}[1]{\oldsubparagraph*{#1}\mbox{}}
  \newcommand{\xxxSubParagraphNoStar}[1]{\oldsubparagraph{#1}\mbox{}}
\fi
\makeatother


\providecommand{\tightlist}{%
  \setlength{\itemsep}{0pt}\setlength{\parskip}{0pt}}\usepackage{longtable,booktabs,array}
\usepackage{calc} % for calculating minipage widths
% Correct order of tables after \paragraph or \subparagraph
\usepackage{etoolbox}
\makeatletter
\patchcmd\longtable{\par}{\if@noskipsec\mbox{}\fi\par}{}{}
\makeatother
% Allow footnotes in longtable head/foot
\IfFileExists{footnotehyper.sty}{\usepackage{footnotehyper}}{\usepackage{footnote}}
\makesavenoteenv{longtable}
\usepackage{graphicx}
\makeatletter
\newsavebox\pandoc@box
\newcommand*\pandocbounded[1]{% scales image to fit in text height/width
  \sbox\pandoc@box{#1}%
  \Gscale@div\@tempa{\textheight}{\dimexpr\ht\pandoc@box+\dp\pandoc@box\relax}%
  \Gscale@div\@tempb{\linewidth}{\wd\pandoc@box}%
  \ifdim\@tempb\p@<\@tempa\p@\let\@tempa\@tempb\fi% select the smaller of both
  \ifdim\@tempa\p@<\p@\scalebox{\@tempa}{\usebox\pandoc@box}%
  \else\usebox{\pandoc@box}%
  \fi%
}
% Set default figure placement to htbp
\def\fps@figure{htbp}
\makeatother

\makeatletter
\@ifpackageloaded{caption}{}{\usepackage{caption}}
\AtBeginDocument{%
\ifdefined\contentsname
  \renewcommand*\contentsname{Table of contents}
\else
  \newcommand\contentsname{Table of contents}
\fi
\ifdefined\listfigurename
  \renewcommand*\listfigurename{List of Figures}
\else
  \newcommand\listfigurename{List of Figures}
\fi
\ifdefined\listtablename
  \renewcommand*\listtablename{List of Tables}
\else
  \newcommand\listtablename{List of Tables}
\fi
\ifdefined\figurename
  \renewcommand*\figurename{Figure}
\else
  \newcommand\figurename{Figure}
\fi
\ifdefined\tablename
  \renewcommand*\tablename{Table}
\else
  \newcommand\tablename{Table}
\fi
}
\@ifpackageloaded{float}{}{\usepackage{float}}
\floatstyle{ruled}
\@ifundefined{c@chapter}{\newfloat{codelisting}{h}{lop}}{\newfloat{codelisting}{h}{lop}[chapter]}
\floatname{codelisting}{Listing}
\newcommand*\listoflistings{\listof{codelisting}{List of Listings}}
\makeatother
\makeatletter
\makeatother
\makeatletter
\@ifpackageloaded{caption}{}{\usepackage{caption}}
\@ifpackageloaded{subcaption}{}{\usepackage{subcaption}}
\makeatother
\journal{Ecology Letters}

\usepackage[]{natbib}
\bibliographystyle{elsarticle-harv}
\usepackage{bookmark}

\IfFileExists{xurl.sty}{\usepackage{xurl}}{} % add URL line breaks if available
\urlstyle{same} % disable monospaced font for URLs
\hypersetup{
  pdftitle={From noise to knowledge: how randomness generates novel phenomena and reveals information},
  pdfauthor={Carl Boettiger},
  pdfkeywords={Coloured noise, demographic noise, environmental
noise, quasi-cycles, stochasticity, tipping points},
  colorlinks=true,
  linkcolor={blue},
  filecolor={Maroon},
  citecolor={Blue},
  urlcolor={Blue},
  pdfcreator={LaTeX via pandoc}}


\setlength{\parindent}{6pt}
\begin{document}

\begin{frontmatter}
\title{From noise to knowledge: how randomness generates novel phenomena
and reveals information}
\author[1]{Carl Boettiger%
\corref{cor1}%
}
 \ead{cboettig@berkeley.edu} 

\affiliation[1]{organization={},addressline={Dept of Environmental
Science, Policy, and Management, University of California Berkeley,
Berkeley CA 94720-3114, USA},postcodesep={}}

\cortext[cor1]{Corresponding author}

        
\begin{abstract}
Noise, as the term itself suggests, is most often seen a nuisance to
ecological insight, a inconvenient reality that must be acknowledged, a
haystack that must be stripped away to reveal the processes of interest
underneath. Yet despite this well-earned reputation, noise is often
interesting in its own right: noise can induce novel phenomena that
could not be understood from some underlying determinstic model alone.
Nor is all noise the same, and close examination of differences in
frequency, color or magnitude can reveal insights that would otherwise
be inaccessible. Yet with each aspect of stochasticity leading to some
new or unexpected behavior, the time is right to move beyond the
familiar refrain of ``everything is important'' (Bjørnstad \& Grenfell
2001). Stochastic phenomena can suggest new ways of inferring process
from pattern, and thus spark more dialog between theory and empirical
perspectives that best advances the field as a whole. I highlight a few
compelling examples, while observing that the study of stochastic
phenomena are only beginning to make this translation into empirical
inference. There are rich opportunities at this interface in the years
ahead.
\end{abstract}





\begin{keyword}
    Coloured noise \sep demographic noise \sep environmental
noise \sep quasi-cycles \sep stochasticity \sep 
    tipping points
\end{keyword}
\end{frontmatter}
    

\section{Introduction: Noise the
nuisance}\label{introduction-noise-the-nuisance}

To many, stochasticity, or more simply, noise, is just that -- something
which obscures patterns we are trying to infer (Knape \& de Valpine
2011); and an ever richer batteries of statistical methods are developed
largely in an attempt to strip away this undesirable randomness to
reveal the patterns beneath (Coulson 2001). Over the past several
decades, literature in stochasticity has transitioned from thinking of
stochasticity in such terms; where noise is a nuisance that obscures the
deterministic skeleton of the underlying mechanisms, to the recognition
that stochasticity can itself be a mechanism for driving many
interesting phenomena (Coulson et al.~2004). Yet this transition from
noise the nuisance to noise the creator of ecological phenomena has had,
with a few notable exceptions, relatively little impact in broader
thinking about stochasticity. One of the most provocative of those
exceptions has turned the classical notion of noise the nuisance on its
head: recognizing that noise driven phenomena can become a tool to
reveal underlying processes: to become noise the informer. Here I argue
that this third shift in perspective offers an opportunity to better
bridge the divide between respective primarily theoretical and primarily
empirical communities by seeing noise not as mathematical curiosity or
statistical bugbear, but as a source for new opportunities for
inference.

In arguing for this shift, it essential to recognize this is a call for
a bigger tent, not for the rejection of previous paradigms. What I will
characterize as `noise the nuisance' reflects a predominately
statistical approach, in which noise, almost by definition, represents
all the processes we are not interested in that create additional
variation which might obscure the pattern of interest. By contrast, an
extensive literature has long explored how noise itself can create
patterns and explain processes from population cycling to coexistence.
These broad categories should be seen as a spectrum and not be mistaken
for either a sharp dichotomy nor a reference to a strictly
empirical-theoretical divide. Each paradigm expands upon rather than
rejects the previous notion of noise: the recognition that noise can
create novel phenomena does not mean that noise cannot also obscure the
signal of some process of interest. Likewise, seeking to use noise as a
novel source of information about underlying processes will be informed
by both previous paradigms, as our discussion will illustrate.

Numerical simulations permit poking and prodding investigation
unencumbered by either experimental design or mathematical formalism.

The code and data for the simulations in this paper are maintained at
\url{https://github.com/cboettig/noise-phenomena}.

To emphasize the underlying trend in the changing roles in which we see
and understand noisy processes, I will also restrict my focus to
relatively simple models primarily from population ecology context.
Simplicity not only makes examples (in equations and in code) more
tractable but also allows us to focus on aspects that are germane to
many contexts rather than unique to particular complexities (Bartlett
1960; Levins 41 1966).

Nevertheless, that complexity matters -- few themes have been better
emphasized in the theoretical literature (Bjørnstad \& Grenfell 2001).
Both the foundational literature and recent research continue to echo
the theme of understanding the impact different real world complexities
have in stochastic dynamics. As such, we will rely on both textbooks and
recent reviews to provide a proper treatment of these issues, and focus
on broader trends.

This review is structured into three sections: Origins of noise,
emergent phenomena, and noise-driven inference. The first section lays
the conceptual groundwork we will need, while also highlighting a shift
to more and more mechanistically rooted descriptions of noise. We will
see where the common formulation of ``deterministic skeleton plus noise
term'' comes from, how it is best justified, and when it breaks down.
The second section introduces noise the creator, showing examples of
ecological phenomena generated by stochasticity. These examples will be
familiar to most specialists but illustrate a different way of thinking
than held by most ecologists, where noise is only a nuisance to be
filtered or averaged out. The third and final section, noise the
informer, turns these examples back-to-front, asking what noise can tell
us about a system: such as its underlying resilience or stability, or
the approach of a catastrophic shift. Examples are fewer here, and have
largely yet to benefit from the introduction of either the rigorous
theorems or more complex models so plentiful in the previous sections.
Yet the promise of prediction, of early warning signs before tipping
points, have spurred broad interest in such noise-based inference. This
review is a call to both deepen the connection to mechanism and better
formalize this thinking, but also look more broadly into other ways in
which noisy phenomena can help inform and predict underlying processes.

\subsection{Demographic stochasticity}\label{demographic-stochasticity}

Demographic stochasticity refers to fluctuations in population sizes or
densities that arise from the fundamentally discrete nature of
individual birth and death events. Demographic stochasticity is a
particularly instructive case for illustrating a mechanism for how noise
arises as an aggregate description from a lower-level mechanistic
process. We summarize the myriad lower-level processes that
mechanistically lead to the event of a `birth' in the population as a
probability: In a population of \(N\) identical individuals at time \(t%
\) a birth occurs with probability \(b_t(N_t)\) (\emph{i.e.} a rate that
can depend on the population size, \(N\)), which increases the
population size to \(N+1\). Similarly, death events occur with
probability \(d_t(N_t)\), decreasing the population size by one
individual, to \(N-1\). Assuming each of these events are independent,
this is a state-dependent Poisson process. The change in the probability
of being in state N is given by the sum over the ways to enter the
state, minus the ways to leave the state: a simple expression of
probability balance known as the master equation (Kampen 2007). Note
that in general this approach is equally applicable to stochastic
transitions of any sort, not just step sizes of +/- 1 and not just birth
and death events, but can include transitions between stage classes or
trait values, including mutations to continuously-valued traits in
evolutionary dynamics (e.g.~Boettiger et al.~2010).

The Gillespie (1977) provides an exact algorithmcfor simulating
demographic stochasticity at an individual level.

The algorithm is a simple and direct implementation of the master
equation, progressing in random step sizes determined by the waiting
time until the next event. Free from both the approximations and
mathematical complexity, the Gillespie algorithm is an interesting
example of where we rely on a numerical implementation to check the
accuracy of an analytic approximation, even in the case of simple models
such as we will discuss. Though the algorithm is often maligned as
numerically demanding, it can be run much more effectively even on large
models on today's computers than when it was first developed in the 70s,
and remains an underutilized approach for writing simple and
approximation-free\footnote{that is, free from the approximation made by
  SDE models as we see in the van Kampen example. All models are, of
  course, only approximations.} stochastic ecological models.

As our objective is to tie the origins of noise more closely to
biological processes, it will be helpful to make the notion of a master
equation concrete with a specific example. We will focus on the classic
case of Levins (1969) patch model, to illustrate the Gillespie algorithm
and the van Kampen system size expansion

\begin{align}
\frac{\mathrm{d} n}{\mathrm{d} t} = \underbrace{c n \left(1 - \frac{n}{N}\right)}_{\textrm{birth}} - \underbrace{e n}_{\textrm{death}}, \label{levins}
\end{align}

where \(n\) individuals compete for a finite number of suitable habitats
\(N\). Individuals die a constant rate \(e\), and produce offspring at a
constant rate \(c\) who then have a probability of colonizing an open
patch that is simply proportional to the fraction of available patches,
\(1 - n/N\).

Figure 1 shows the results of two exact SSA simulations of the classic
patch model of Levins (1969).

\section{Conclusions}\label{conclusions}

This review has explored three paradigms in how noise is viewed
throughout the ecological literature, which I have dubbed respectively:
noise the nuisance, noise the creator, and noise the informer. Noise can
be seen as a nuisance almost by definition: in examining the origins of
noise, we have seen how stochasticity is introduced not because
ecological processes are random in some fundamental sense, but rather,
because those processes are influenced by a complex combination of
forces we do not model explicitly. In this view, noise captures all that
additional variation that is separate from the process of interest, and
a rich array of statistical methods allow us to separate the one from
the other in observations and experiments. By examining the origins of
noise, we have seen that despite the complex ways in this noise can
enter a model, that a Gaussian white-noise approximation (Kampen 2007;
Black \& McKane 2012) is often appropriate given a limit of a large
system size -- a fact often invokedn implicitly but rarely derived
explicitly from the theorems of Kurtz (1978) and others.

In this context, noise does not act to create phenomena of interest
directly. The sudden transitions we seek to anticipate are still
explained by the deterministic part of the model -- bifurcations. But
nor is noise a nuisance that merely cloaks this deterministic skeleton
from plain view: rather, it becomes a novel source of information that
would be inaccessible from a purely deterministic approach. I believe
more examples of how noise can inform on underlying processes is
possible, but will require greater dialog between these world views.

\section{Acknowledgements}\label{acknowledgements}

The author acknowledges feedback and advice from the editor, Tim Coulson
and two anonymous reviewers. This work was supported in part by USDA
National Institute of Food and Agriculture, Hatch project
CA-B-INS-0162-H.


\renewcommand\refname{References}
  \bibliography{bibliography.bib}



\end{document}
